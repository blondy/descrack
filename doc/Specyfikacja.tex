\documentclass[a4paper,11pt]{article}
\usepackage[T1]{fontenc}
\usepackage[polish]{babel}
\usepackage[utf8]{inputenc}
\usepackage{lmodern}
\usepackage{hyperref}
\usepackage[top=2cm, bottom=2cm, left=2cm, right=2cm]{geometry}

\title{ \sc{Programowanie Równoległe i Rozproszone} \\
\emph{Równoległe wyznaczanie tęczowych tablic w zagadnieniach kryptografii dla haseł zaszyfrowanych algorytmem DES} }

\author{Sławomir Blatkiewicz
\and Piotr Piechal
\and Michał Zochniak}

\begin{document}

\maketitle
\tableofcontents

\section{Specyfikacja funkcjonalna}

\subsection{Opis problemu}
Rainbow tables używane są do przyspieszenia łamania haseł zapisanych w systemach informatycznych. Ponieważ autentykacja wiąże się z koniecznością przechowywania haseł w systemie, nie można ich przechowywać w sposób jawny, bo każdy kto by uzyskał dostęp do systemu miałby dostęp do haseł wszystkich użytkowników. Dlatego hasła przechowywane są w postaci wartości funkcji skrótu.

Aby złamać to zabezpieczenie stosuje się ataki typu brute force. Polegają one na sprawdzeniu wszystkich możliwych kombinacji w poszukwianiu tej właściwej. Jednak przy długich hasłach złożoność tego procesu staje się ogromna. Właśnie z pomocą przychodzą Rainbow tables.

Pozwalają one zmniejszyć w znaczący sposób czas potrzebny na odszukanie prawidłowego hasła algorytmem brute force jendak za cenę dużej złożoności pamięciowej. A to za sprawą konieczności przechowywania dużej ilości par: hasło, hash.

Lamanie hasel jest uzywane miedzy innymi do testow penetracyjnych systemu. Majac skróty haseł systemu mozemy sprobowac znalesc słabe hasla uzytkownikow ktore sa potencjalnym niebezpieczenstwem systemu.

Zajmować będziemy się funkcją mieszającą zaimplementowaną przy uzyciu DESa w funkcji bibliotecznej crypt(3).
Funkcja ta jest uzywana np. przy zarzadzaniu haslami w systemach *nixowych. Mimo, ze wiekszosc administratorow konfiguruje swoje systemy aby uzywaly bezpieczniejszych funkcji skrotu, wiele systemow wciaz uzywa crypt(3) z DESem.


\subsection{Rainbow tables}
Tęczowe tablice wykorzystują funkcje hashujące oraz funkcje redukujące. Funkcje hashujące generują skróty(hashes) na podstawie haseł(plaintexts). Hashe są funkcją jednokierunkową, czyli na podstawie hasha bardzo trudno jest wygenerować z powrotem plaintext. Reduction function z kolei nie są wcale funkcjami odwracalnymi dla wygenerowanych hashów. Mapują hashe na tekst jawny, czyli na podstawie hasha funkcja redukująca tworzy plaintext już inny od wejściowego plaintexta, ale w dalszym ciągu istniejący w przestrzeni haseł. 

Łańcuchy tworzące tęczową tablicę są łańcuchami hashy oraz funkcji redukujących rozpoczynających się od danego tekstu jawnego i kończących się pewnym skrótem. Łańcuch w tęczowej tablicy rozpoczyna się pewnym tekstem jawnym(hasłem), wylicza z niego hash, redukuje hash do innego tekstu jawnego, wylicza skrót na tym nowym haśle, redukuje ten skrót itd.

Ogromnie zmniejsza to narzut pamieciowy wygenerowanej tablicy, poniewaz tablica zawiera jedynie początkowy tekst jawny oraz hash dla którego zdecydowano zakończyć generowanie łańcucha(będący końcem łańcucha).

Mając dany hash dla nieznanego hasła, próbujemy znaleść ten skrót w w którymś z wygenerownych łańcuchów tablicy.

Algorytm wyszukiwania hasha:
\begin{itemize}
  \item Szukanie skrótu w liście skrótów kończących łańcuch, jeżeli istnieje to zakończ
  \item Jeżeli nie istnieje, to zredukuj dany skrót do innego hasła, a następnie oblicz skrót z nowego hasła
  \item Idź do kroku pierwszego
  \item Jeżeli skrót pasuje do jednego z finalnych skrótów, łańcuch dla którego skrót pasuje z finalnym skrótemm zawiera oryginalny skrót
\end{itemize}

Niedoskonałosci:
\begin{itemize}
  \item Kolizje: Wtedy gdy dany hash jest generowany przez różne hasła. Powoduje to problem, ponieważ tworzone są łańcuchy zaczynające się od różnych haseł, a zbiegające się do tego samego hasha.
  \item Pętle: Pojawiające się wtedy kiedy hash jest redukowany do hasła, które było wcześniej hashowane w łańcuchu. To hasło jest znowu hashowane, następnie ponowna redukcja do tego samego hasła i tak w kółko.
\end{itemize}

\subsection{Zaproponowane rozwiązanie}
Tablice tęczowe będą generowane równolegle. System zostanie zrealizowany przy pomocy technologii Interfejsu Przekazywania Wiadomości MPI, z użyciem implementacji MPICH2.

Tworzenie tablic będzie w oparciu o jeden proces Zarządcę oraz dowolna ilość procesów wykonawczych. 

Tablice będą generowane dla określonej dziedziny, czyli zakresu haseł (słownika) które ma pokryć tablica. Proces zarządcy będzie dokonywał dekompozycji domenowej, tj. podzieli dziedzinę haseł równomiernie na procesy wykonawcze. Procesy wykonawcze będą tworzyć łańcuchy skrótów i odsyłać je do zarządcy.

Odpytywanie tablicy (dopasowanie tekstu jawnego do podanego skrótu) będzie robione za pomocą oddzielnej aplikacji, która będzie przeszukiwać wygenerowane tablice.

\end{document}
